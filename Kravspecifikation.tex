\documentclass[oneside]{memoir}
\setcounter{tocdepth}{2}
\setcounter{secnumdepth}{1}
\usepackage[utf8]{inputenc}
\usepackage{amsmath}
\usepackage{amsfonts}
\usepackage{amssymb}
\usepackage{lipsum}
\usepackage{hyperref}
\usepackage{graphicx}
\usepackage{multirow}
\usepackage{float}
\usepackage{enumitem}
\usepackage{url}
\usepackage[width=15.00cm]{geometry}
\title{Kravspecifikation}
\author{Gruppe 7}
\begin{document}
		\maketitle
	\textbf{Deltagere i afleveringen}
			\begin{table}[H]
		\centering
		\label{my-label}
		\begin{tabular}{|l|l|l|}
			\hline
			Studienummer & Navn                   & Studieretning \\ \hline
			201409553    & Alex Imhof - \textbf{Kontaktperson } &E\\ \hline
			201500156    & Simon Møller Hermansen & IKT           \\ \hline
			201409557    & Victor Busk            & IKT           \\ \hline
			201409552    & Andreas Lüdemann       & IKT           \\ \hline
			20115758     & Ole Clausen            & IKT           \\ \hline
			201407641    & Christoffer Søby Høj   & IKT           \\ \hline
			201409551    & Daniella Tola          & E             \\ \hline
		\end{tabular}
	\end{table}
\begin{flushleft}
		\textbf{Vejleder: } Carl Jakobsen
\end{flushleft}
\begin{table}[]
	\centering
	\label{my-label}
	\begin{tabular}{|l|l|l|l|}
		\hline
		Ver. & Dato     & Initialer & Beskrivelse    \\ \hline
		0.01 & 02.10.15 & DT        & Første version \\ \hline
		0.02 & 08.10.15 & OC		& Rettet efter 1. Review \\ \hline
		0.03 & 29.10.15 & OC 		& Rettelser med inputs fra CJ \\ \hline
		0.04 & 07.11.15 & DT		& Rettelser og tilføjet beskrivelse af projektudvikling \\ \hline
		0.05 & 24.11.15 & DT		& Rettelser efter 2. Review \\ \hline
	\end{tabular}
\end{table}

	\newpage
		\tableofcontents
		\chapter{Kravspecifikation}
		Formålet med dette dokument er at beskrive hvilke funktionelle og ikke funktionelle krav, der er bestemt for at lave produktet.
Specifikation af systemet beskrives også med krav, der fortæller hvad systemets funktioner er, og om
systemets kvalitet.
		\section{Systembeskrivelse}	
Vi har valgt at lave et lyskontrolsystem (Light Control System). Dette system består af et brugerinterface (GUI) på en pc og en elektronisk kodelås (padlock) der kan styre en RGB-LED. Et led af kommunikationen mellem styringen og RGB-LED’en foregår over et vekselspændingsnet. 
Kommunikationen over nettet sker mellem to stykker hardware der fungerer som sender (transmitter) og modtager (receiver). 
Brugeren kan på padlocken indtaste to koder ved at trykke på en knap (lock/unlock Button). Såfremt adgangskoderne er korrekte, vil de indstillinger man har indtastet på pc’en blive sendt videre til receiveren og kan på den måde påvirke RGB-LED’en. Hvis man derimod ikke har indtastet koderne korrekt vil indtastningerne på pc’en ikke blive sendt videre. På pc’en kan man tænde og slukke for en RGB-LED’en samt skrue op og ned for intensiteten af de enkelte farver.


\begin{figure}[H]
\centering
\includegraphics[width=0.9\linewidth]{"Kravspecifikation/Systembeskrivelse"}
\caption{Systemet med brugergrænseflade og enheder}
\label{fig:Systembeskrivelse}
\end{figure}

        \section{Funktionelle krav formuleret ved UCs}
    \subsection{Use Case Diagram}
\begin{figure}[H]
\centering
\includegraphics[width=0.9\linewidth]{"Kravspecifikation/ucdiagram"}
\caption{Use Case diagram for Light Control System}
\label{fig:ucdiagram}
\end{figure}

        \subsection{Aktørbeskrivelse}
Aktørerne i use case diagrammet beskrives følgende: 
\newline
 \textbf{User:} Det er den primære aktør, der har muligheden for at styre lysintensiteten, skifte farve på lyset, tænde og slukke lyset. Samt at låse systemet.
\textbf{RGB-LED:} Det er den sekundære aktør, som kontrolleres igennem brugergrænsefladen.
        \subsection{Fully Dressed Use Cases}
The RGB-LED is an X.10 compatible device.


\begin{table}[]
\centering
\caption{UC1}
\label{my-label}
\begin{tabularx}{\textwidth}{|l X|} 
\hline
Name                             & UC1: Regulate Light Intensity of RGB-LED   \\ \hline
Goal                             & Control RGB-LED variables from computer    \\ \hline
Intitiation                      & User mouse over sliders and click          \\ \hline
Actors                           & User, RGB-LED                              \\ \hline
Number of concurrent occurrences & 1                                          \\ \hline
Precondition                     & System is operational. System is unlocked. \\ \hline
Post condition                   & RGB-LED behaves as prompted by the faders  \\ \hline
Main Scenario                    & 
                                 
                                    \begin{enumerate} 
                                 \item User chooses to increase/decrease light intensity of the various colors on GUI 
                              \begin{enumerate}[label*=\arabic*.]
                                \item User pulls “Red” slider up, down or changes nothing
                                \item User pulls “Green” slider up, down or changes nothing
                                \item User pulls “Blue” slider up, down or changes nothing
                                \item User adjusts the “Color Wheel”
                                        \end{enumerate}
                                \item User interface updates color preview
                                \item Computer sends data to transmitter
                                \item Transmitter sends commands to receiver
                                \item Receiver displays received color data on RGB-LED
                                 \end{enumerate} 
                                 \\     \hline
\end{tabularx}
\end{table}



\begin{table}[]
\centering
\caption{UC2}
\label{my-label}
\begin{tabularx}{\textwidth}{|l X|}
\hline
Name                             & UC2: Turn on/off RGB-LED   \\ \hline
Goal                             & To turn the RGB-LED on/off    \\ \hline
Intitiation                      & User mouse over and click button          \\ \hline
Actors                           & User, RGB-LED                              \\ \hline
Number of concurrent occurrences & 1                                          \\ \hline
Precondition                     & Computer is on. Control program is open and the desired device
has been chosen. System is unlocked. \\ \hline
Post condition                   & RGB-LED is turned on with chosen settings  \\ \hline
Main Scenario                    & \begin{enumerate}
                                    \item User mouse over and clicks the "on/off" button
                                    \item User interface stops sending data to transmitter system
                                    \end{enumerate} \\ \hline
                                  
\end{tabularx}
\end{table}


\begin{table}[]
\centering
\caption{UC3}
\label{my-label}
\begin{tabularx}{\textwidth}{|l X|}
\hline
Name                             & UC3: Unlock Transmitter   \\ \hline
Goal                             & Unlock Transmitter from padlock   \\ \hline
Intitiation                      & User moves switches          \\ \hline
Actors                           & User                              \\ \hline
Number of concurrent occurrences & 1                                          \\ \hline
Precondition                     & Transmitter is locked and display shows the text “Locked”. \\ \hline
Post condition                   & Transmitter is unlocked and display shows the text: “Unlocked”.  \\ \hline
Main Scenario                    & \begin{enumerate}
                                    \item User enters a 4-bit password by moving the switches on
the padlock
                                    \item User presses unlock button on padlock
                                    \item Padlock validates password
                                    \newline Extension 1: The password is wrong
                                    \item Display shows text: “Unlocked”
                                    \item Padlock sends an unlock signal (5 V) to transmitter
                                    \end{enumerate} \\ \hline
 Extension                      & [Extension 1: The password is wrong]
                                    \newline \begin{enumerate}
                                    \item Display continues displaying: "Locked"
                                    \end{enumerate} \\ \hline
\end{tabularx}
\end{table}


\begin{table}[]
\centering
\caption{UC4}
\label{my-label}
\begin{tabularx}{\textwidth}{|l X|}
\hline
Name                             & UC4: Lock Transmitter   \\ \hline
Goal                             & Lock Transmitter   \\ \hline
Intitiation                      & User presses lock button         \\ \hline
Actors                           & User                              \\ \hline
Number of concurrent occurrences & 1                                          \\ \hline
Precondition                     & Transmitter is unlocked and display shows text: “Unlocked” \\ \hline
Post condition                   & Transmitter is locked and display says: “Locked”  \\ \hline
Main Scenario                    & \begin{enumerate}
                                    \item User presses lock button

                                    \item Display shows text: "Locked"
                                    
                                    
                                    \item The padlock sends lock signal (0V) to transmitter
                                    \end{enumerate} \\ \hline

\end{tabularx}
\end{table}


\section{Kvalitetskrav/ikke-funktionelle krav}
\begin{enumerate}
\item Generelle Krav
    \begin{enumerate}[label*=\arabic*.]
    \item Den benyttede AC net skal af sikkerhedsmæssige årsager være 18V rms, 560mA.
    \item RS232 benyttes til at lave seriel kommunikation. Baud raten skal være større end 109 Baud, og
mindre end 115201 Baud.
    \end{enumerate}
\item RGB-LED'en
    \begin{enumerate}[label*=\arabic*.]
    \item Lysintensiteten fra RGB-LED’en skal kunne dimmes i 8 trin pr. farve (R, G, B) via. X.10 standart
kommandoer.
    \item RGB-LED’en skal fade mellem disse 8 trin.
    \item Hver farve i RGB-LED’en skal trække en strøm på 20 mA +/- 1 mA\footnote{\url{https://en.wikipedia.org/wiki/Visible_spectrum}} .
    \item RGB-LED’en skal kunne udsende 3 forskellige bølgelængde: 620-750 nm(rød), 495-570 nm(grøn),
450-495 nm(blå)\footnote{\url{http://www.dioder-online.dk/rgb-diode-9400mcd-35-5mm-ultrabright.html}}.
    \end{enumerate}
\end{enumerate}

En del af kravene er, at vi følger X10 standarden. Se evt. side 13\footnote{\url{http://ww1.microchip.com/downloads/en/AppNotes/00236a.pdf}}. Vi benytter os af extended code og
derefter sender en selvlavet funktion, og derfor kan man sige, at vi følger X10 standarden delvist.


\section{Andre krav}
\subsection{Krav til anvendt udviklingsproces}
ektet bliver udviklet ved at bruge Scrum princippet delvist. Gruppen har planlagt, at der er møde en gang
om ugen, hvor hver eneste gruppemedlem fortæller om sin status med udviklingen. Der benyttes også sprint,
hvor der er interne deadlines for hver gruppemedlem. Gruppen har delt en gruppelederrolle ud til Alex, hvor
han indkalder til møder og har det samlede overblik. ASE-modellen er den udviklingsmodel, der blvier brugt i dette projekt.

\subsection{Krav til anvendte udviklingsværktøjer}
Det er blevet besluttet at den grafiske brugergrænseflade GUI udvikles i softwaren Qt. Øvrige software-dele
i projektet skrives i C++.








\begin{table}[]
\centering
\caption{UC1}
\label{my-label}
\begin{tabularx}{1.1\textwidth}{|X|X|} 
\hline
Name                             & UC1: Regulate Light Intensity of RGB-LED   \\ \hline
Goal                             & Control RGB-LED variables from computer    \\ \hline
Intitiation                      & User mouse over sliders and click          \\ \hline
Actors                           & User, RGB-LED                              \\ \hline
Number of concurrent occurrences & 1                                          \\ \hline
Precondition                     & System is operational. System is unlocked. \\ \hline
Post condition                   & RGB-LED behaves as prompted by the faders  \\ \hline
Main Scenario                    &   
                                  \begin{enumerate} 
                                 \item User chooses to increase/decrease light intensity of the various colors on GUI 
                              \begin{enumerate}[label*=\arabic*.]
                                \item User pulls “Red” slider up, down or changes nothing
                                \item User pulls “Green” slider up, down or changes nothing
                                \item User pulls “Blue” slider up, down or changes nothing
                                \item User adjusts the “Color Wheel”
                                        \end{enumerate}
                                \item User interface updates color preview
                                \item Computer sends data to transmitter
                                \item Transmitter sends commands to receiver
                                \item Receiver displays received color data on RGB-LED
                                 \end{enumerate} 
                                 
                                 \\     \hline
\end{tabularx}
\end{table}


\end {document}