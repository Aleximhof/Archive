\documentclass[oneside]{memoir}
\setcounter{tocdepth}{2}
\setcounter{secnumdepth}{1}
\usepackage[utf8]{inputenc}
\usepackage{amsmath}
\usepackage{amsfonts}
\usepackage{amssymb}
\usepackage{lipsum}
\usepackage{hyperref}
\usepackage{graphicx}
\usepackage{float}
\usepackage[width=15.00cm]{geometry}
\title{Projektformulering}
\author{Gruppe 7}
\begin{document}
		\maketitle
	\textbf{Deltagere i afleveringen}
			\begin{table}[H]
		\centering
		\label{my-label}
		\begin{tabular}{|l|l|l|}
			\hline
			Studienummer & Navn                   & Studieretning \\ \hline
			201409553    & Alex Imhof - \textbf{Kontaktperson } &E\\ \hline
			201500156    & Simon Møller Hermansen & IKT           \\ \hline
			201409557    & Victor Busk            & IKT           \\ \hline
			201409552    & Andreas Lüdemann       & IKT           \\ \hline
			20115758     & Ole Clausen            & IKT           \\ \hline
			201407641    & Christoffer Søby Høj   & IKT           \\ \hline
			201409551    & Daniella Tola          & E             \\ \hline
		\end{tabular}
	\end{table}
\begin{flushleft}
		\textbf{Vejleder: } Carl Jakobsen
\end{flushleft}

\begin{table}[]
	\centering
	\label{my-label}
	\begin{tabular}{|l|l|l|l|}
		\hline
		Ver. & Dato     & Initialer & Beskrivelse    \\ \hline
		0.01 & 02.10.15 & OC        & Første version \\ \hline
		0.02 & 08.10.15 & CSH		& Rettet efter 1. Review \\ \hline
		0.03 & 24.11.15 & VB		& Rettelser efter 2. Review \\ \hline
	\end{tabular}
\end{table}

	\newpage
		\tableofcontents
		
\chapter{Projektformulering}
\section{Problemformulering}
\begin{flushleft}
Opgaven i dette projekt er at udvikle et lysstyrings system. For at brugeren kan styre en RGB led, implementeres systemet med en pc med en grafisk brugerflade, der via en microcontroller kan kommunikere med en RGB-LED gennem X.10-protokollen over el-nettet. Senderen aktiveres/deaktiveres med en elektronisk kodelås. 
For at der kan simuleres aktivitet, bygges en X.10 modtagerenhed, bestående af en microcontroller hvorpå der kan tilføjes forskellige funktionaliteter, i vores system kontrolleres en RGB-LED. Den skal kunne styres fra systemet via computeren på lige fod med andre X.10 kompatible modtagerenheder. 
Systemet skal altså kunne:
\begin{itemize}
	\item Aktivere/deaktivere transmitteren fra en elektronisk kodelås
	\item Konfigureres fra en computer
	\item Styre X.10 enheder på el-nettet (18 V AC)
	\item Tænde, slukke og regulere farve og lysintensiteten på en RGB-LED via et grafisk brugerinterface.
	\item Tilføje flere enheder (RGB-LED'er)
\end{itemize}
Visionen bag projektet er at skabe et system, der på en intuitiv måde kan interagere med X.10 enheder på el-nettet. 
Målet med projektet er at udfærdige et system hvor alle basis funktioner i X.10 protokollen er implementerede således at en funktionel prototype af systemet, X.10 sender og modtager, via styring fra en grafisk brugerflade på en pc kan stå for hhv. at sende X.10 kommandoer ud på el-nettet samt filtrere disse signaler fra el-nettet og fortolke disse til styring af en microcontroller. Desuden skal der kunne tilføjes flere enheder (RGB-LED'er) i systemet. Der bliver i dette projekt dog kun fokuseret på styring af en RGB-LED.
\end{flushleft}
\newpage
\section{Problemafgrænsning}
\begin{flushleft}
I projektet vil vi udvikle et system der som udgangspunkt begrænser sig til at indeholde:
\begin{itemize}
    \item Kodelås på DE2-Board
    \item Brugerinteraktion via computer
    \item Seriel kommunikation mellem computer og STK500(X.10 kontroller) 
    \item X.10 interface – en sender enhed og en modtager 
    \item X.10 modtager enhed med microcontroller til styring af en lysintensitet på en RGB-LED.
\end{itemize}
Der ligges i projektet især vægt på arbejdet med systemdesignet. Herunder dokumentation af produktudviklingen. Det primære fokus vil være at en funktionsdygtig X.10 protokol designes. Det vil yderligere efterstræbes at denne implementeres i en funktionel prototype, hvor det er muligt for en bruger at kunne sende data til en enhed gennem el nettet via interaktion med computeren. Sendte data skal kunne fortolkes af en microcontroller til regulering af lysintensitet på en RGB-LED. 
\newline 
\newline 
Af sikkerhedsmæssige grunde udvikles systemet på et lukket 18 Volts AC net.
\end{flushleft}
\end{document}

